\documentclass[10pt]{article}
\usepackage[margin=0.75in]{geometry}   
\usepackage{subcaption}       
\usepackage{graphicx}
\usepackage{fontspec}
%\setmainfont[Ligatures=TeX]{Helvetica}
\newfontfamily\myfont{Courier}
\usepackage{amsthm, amsmath, amssymb}
\usepackage{setspace}\onehalfspacing
\usepackage[loose,nice]{units} %replace "nice" by "ugly" for units in upright fractions
\usepackage[arrowdel]{physics}
\usepackage{bbold}
\usepackage{simplewick}
\usepackage[compat=1.0.0]{tikz-feynman}
\newcommand\tab[1][1cm]{\hspace*{#1}}
\DeclareRobustCommand{\rchi}{{\mathpalette\irchi\relax}}
\newcommand{\irchi}[2]{\raisebox{\depth}{$#1\chi$}} % inner command, used by \rchi
\DeclareRobustCommand{\rgamma}{{\mathpalette\irgamma\relax}}
\newcommand{\irgamma}[2]{\raisebox{\depth}{$#1\gamma$}}
\usepackage{makecell}
\newcolumntype{\vl}{!{\vrule width 1pt}}
\usepackage{float}
\usepackage{breqn}
\usepackage{courier}


%% The font package uses mweights.sty which has some issues with the
%% \normalfont command. The following two lines fixes this issue.
%\let\oldnormalfont\normalfont
%\def\normalfont{\oldnormalfont\mdseries}


\title{Kinematic Factors}
\author{Archana Radhakrishnan}
\date{Spring 2019}

\begin{document}
\maketitle



\section{Running the code} 
\textbf{Requirements:} \textbf{{\myfont adat, eigen}}\\
\textbf{Running:} The main code is \textbf{{\myfont compute_matrix_prefactor}}. The input xml contains:
\begin{enumerate}
 \item the $J^{PC}$ and the L
 \item the mass square
 \item the maximum mom 
 of the source, the sink and the current.
 \item the minimum number of time slices to use in the fit
\end{enumerate}
From this you can create an xml say, {\myfont compute_matrix_prefactor.out.xml} ,  that contains all the non-zero kinematic factors. In order to make an input xml for `redstar' one can use {\myfont compute_matrix_prefactor.out.xml} and another xml {\myfont gen_redstar.ini.xml} as an input and run \textbf{{\myfont gen_redstar_xml}}. This will produce the output in the form that can used by `redstar'. The examples of the input xml are given in 

\begin{center}
{\myfont Kinematic_factor/compute_Kfactors/build/runs}. 
\end{center}

\par
One can add other form factors to the code and compute them as well. As of now it only supports vector to pseudo-scalar transitions with a vector current.
\section{Form Factors} 
For composite particles, like hadrons, the interaction to an external current is through the individual constituents. Thus, for the interaction of a hadron to a photon requires the perturbative expansion in terms of quarks. But this is not possible as the quarks interact through strong interactions and a perturbative expansion is not possible in this regime. One can explain the interactions through a pehomenological quantity called the \textbf{form-factor}.\par
Ideally, the structure of the hadrons from QCD should give the form of the form-factors but in the absence of this understanding, we rely on symmetries of QCD to constrain the form-factors. We apply this for a vector to pseudo-scalar transition. \par


\subsection{Vector to Pseudo-scalar Form Factor}
Consider a process $\bra{PS(p_f)} j^{\mu}(p_i - p_f)\ket{V(p_i, \lambda_i)}$. The independent vectors at our disposal from this process are

\begin{center}
$p_f$, $p_i$ and for the vector a polarization $\epsilon(p_i, \lambda_i)$ and $\epsilon^{\mu\nu\rho\sigma}$
\end{center}

The scalars are ($\epsilon\cdot p_i = 0$ and $\epsilon^2 = 0/1$):

\begin{center}
$p_f^2$, $p_i^2$, $p_f^{\mu}p_{i\mu}$,  $p_{f}^{\mu}\epsilon_{\mu}$
\end{center}

The three-point function must be linear in $\epsilon(p_i, \lambda_i)$ using a basis of $(p_i+p_f)$ and $(p_i-p_j)$. The total number of possible form-factors that preserve the Lorentz structure are

\begin{align*}
\bra{PS(p_f)} j^{\mu}(p_i - p_f)\ket{V(p_i, \lambda_i)} = \epsilon^{\mu\nu\rho\sigma}(p_i + p_f)_{\nu}(p_i - p_f)_{\rho}\epsilon_{\sigma}(p_i, \lambda_i)F_1(Q^{2}) + \\ (p_i + p_f)^{\nu}(p_i - p_f)_{\nu}\epsilon^{\mu}(p_i, \lambda_i)F_2(Q^{2})+\\  (p_i - p_f)^{\mu}(p_i + p_f)_{\nu}\epsilon^{\nu}(p_i, \lambda_i))F_3(Q^{2})\\ + (p_i + p_f)^{\mu}(p_i - p_f)_{\nu}\epsilon^{\nu}(p_i, \lambda_i)F_4(Q^{2})
\end{align*}

Applying invariance under parity,

\begin{align*}
\bra{PS(p_f)} j^{\mu}(p_i - p_f)\ket{V(p_i, \lambda_i)} = \bra{PS(p_f)} P^{-1}P j^{\mu}(p_i - p_f) P^{-1}P \ket{V(p_i, \lambda_i)} \\ = (P)^{\mu}_{\nu}(-\bra{PS(-p_f)}) j^{\nu}( p_i - p_f)(-\ket{V(-p_i, \lambda_i)})
\end{align*}

Under parity

\begin{center}
$p_\mu \rightarrow -p_\mu$ and $\epsilon_{\mu}(p,\lambda) \rightarrow -P^{\nu}_{\mu}\epsilon_{\nu}(-p,\lambda)$
\end{center}
And thus the factors vanish. \par
Applying current conservation

\begin{align*}
\partial_{\mu}\bra{PS(p_f)} j^{\mu}(p_i - p_f)\ket{V(p_i, \lambda_i)} = (p_i - p_f)_{\mu}\bra{PS(p_f)} j^{\mu}(p_i - p_f)\ket{V(p_i, \lambda_i)} = 0
\end{align*}

And thus, only $F_1(Q^2)$ remains \cite{shultz}.





\section{The Vector to Pseudo-scalar Form Factor on the Lattice } 
The transition from a vector to pseudo-scalar in the presence of a vector current has only one form factor.
The vector to pseudo-scalar form factor for $\bra{PS(p_f)} j^{\mu}(p_i - p_f)\ket{V(p_i, \lambda_i)}$is of the form,

\begin{equation}
\label{ff}
\epsilon^{\mu\nu\rho\sigma}(p_i + p_f)_{\nu}(p_i - p_f)_{\rho}\epsilon_{\sigma}(p_i, \lambda_i)F(Q^{2})
\end{equation}

\par
On the lattice, we often calculate vectors in the circular basis. To convert the current from the vector to the circular basis using the convention in \cite{hel}, we multiply by,  $i \epsilon^{*}_{\mu}(0,m)$, then to change to $\lambda$ basis from $m$ basis we multiply by the Wigner-D matrix \cite{hel}.
\begin{align*}
O(J, \lambda_{\gamma}, \vec{p}) = \sum_{m} D^{*J}_{m\lambda_{\gamma}}(R_{\gamma}) O(J, m, p_z) \hspace{5mm} R_{\gamma} = Rotation_{p_z \rightarrow \vec{p}}  \\
O(1,m) = i \epsilon^{*}_{\mu}(0,m)O^{\mu} \hspace{5mm} m = \pm 1, 0
\end{align*}

Applying the above equation gives $ D^{1}_{m\lambda_{\gamma}}(R_{\gamma})i \epsilon^{*}_{\mu} (0, m)\bra{PS(p_f)} j^{\mu}(p_i - p_f)\ket{V(p_i, \lambda_i)}$ and we can write the pre-factor of  $F(Q^2)$ from (Eq: \ref{ff}) as

\begin{equation}
\label{amp}
\epsilon^{\mu\nu\rho\sigma}  D^{1}_{m\lambda_{\gamma}}(R_{\gamma}) i \epsilon^{*}_{\mu}(0, m)\epsilon_{\sigma}(p_i, \lambda)(p_i + p_f)_{\nu}(p_i - p_f)_{\rho}
\end{equation}

\par
In the continuum, since it is trivial to move from $J,M$ basis to helicity basis at rest. We can generalize helicity basis to all particles be it in motion or at rest. \par
On the lattice Lorentz symmetry is broken and we deal with the irreducible representations of the cubic group. Thus, the initial state, the final state and the current can be represented by states with particular angular momentum and parity projected into irreps of the orthogonal group in the case of particles at rest and states of particular helicity and parity projected into irreps of the Dirac group in the case of particles in motion. In short, instead of states with particular helicity and angular momentum we have states projected into particular `irreps' and `irrep' rows. \par 
From \cite{hel} any state in helicity basis with helicity $\lambda$, momentum p, angular momentum J and parity P can be projected into a particular irrep by:

\begin{equation}
\label{hl}
\ket{p,\lambda}^{J,P} = O^{\dagger J,P}(p,\lambda)\ket{0} \xrightarrow[\Lambda, \mu]{subduce} \sum_{\lambda = \pm|\lambda| } S^{ \Lambda,\tilde{\eta}  }_{\mu\lambda} O^{\dagger J,P}(p,\lambda)\ket{0}
\end{equation}

here $\tilde{\eta}$ is $P(-1)^{J}$, $\Lambda$ is the irrep and $\mu$ is the row of the irrep the state is subduced into. Using this transformation we can move from the continuum basis to the irreps of cubic group that we deal with. \par
Applying (Eq: \ref{hl}) in (Eq: \ref{amp}), when the source is in irrep(row) $\Lambda_i (\mu_i) $, the sink in $\Lambda_f (\mu_f) $ and the current in $\Lambda_\gamma (\mu_\gamma) $ we get the pre-factor of $F(Q^2)$ to be

\begin{equation}
\label{fnl}
\sum_{\lambda_{f} = \pm|\lambda_{f}| ,\lambda_{\gamma} = \pm|\lambda_{\gamma}| , \lambda_i = \pm|\lambda_i| } i S^{ \Lambda_f,\tilde{\eta}_f  }_{\mu_f 0}  S^{ \Lambda_{\gamma},\tilde{\eta}_{\gamma}  }_{\mu_{\gamma} \lambda_{\gamma} } S^{ \Lambda_i,\tilde{\eta}_i  }_{\mu_i\lambda_i}\epsilon^{\mu\nu\rho\sigma}\sum_{m}D^{1*}_{m\lambda_{\gamma}}(R_\gamma)\epsilon^{*}_{\mu}(0, m)\epsilon_{\sigma}(p_i, \lambda)(p_i + p_f)_{\nu}(p_i - p_f)_{\rho}
\end{equation}

We calculate (Eq: \ref{fnl}) for each source, sink and current irrep(row) to get the pre-factors and divide it out to get the form-factors.


\section{Calculation of the kinematic factors for $PS \rightarrow V$ transitions}
The algorithm followed by the code:
\begin{itemize}
	\item To get the kinematic factors for the process and for a given maximum momentum at the source and the sink, we first loop over all possible momenta at the source and sink.
	\item The $p_0$ at the source and sink is calculated by the relation $p_0 = \sqrt{m^2 + \Big(\frac{2\pi}{L\xi} \vec{p}\Big)^2}$. The four momentum convention used is $(+1 -1 -1 -1)$
	\item The four momentum of the insertion is determined based on conservation. From this the virtual mass of the current can also be determined $m_{\gamma} = \sqrt{E_{\gamma}^2 - \Big(\frac{2\pi}{L\xi} \vec{p_{\gamma}} \Big)^2} $
	\item Next one has to determine the Euler angles for the allowed lattice rotations that take the momentum from $p_z$ to the given direction at source, sink and insertion. It is $R_{lat}R_{ref}$ based on the conventions in \cite{hel}, there is a list of the Euler angles for the rotations in {\myfont adat/irrep/irrep_utils} to which the code is interfaced. It is trivial to convert the 3-rotation matrix to 4, since the spatial rotations do not change the $0^{th}$ component just add (1 0 0 0) (Table: \ref{ref_rot})

\begin{table}[!htbp]
\centering
\hspace{9mm}
\begin{minipage}{0.5\linewidth}
\begin{tabular}{c|c|c|c|c}
\centering
Little group & $\vec{p}_{ref}$ & $\phi$ & $\theta$ & $\psi$ \\ 
\hline
$\mbox{Dic}_4$ & $(0,0,n)$ & 0 & 0 & 0\\
$\mbox{Dic}_2$ & $(0,n,n)$ & $\pi/2$& $\pi/4$ & $-\pi/2$ \\
$\mbox{Dic}_3$ & $(n,n,n)$ & $\pi/4$ & cos$^{-1}(1/\sqrt{3})$ & 0\\
\end{tabular}
\end{minipage}
\caption{Rotations, $R_{ref}$ , used, as described in the text, for rotation $\hat{R}_{\phi,\theta,\psi} = e^{-i\phi\hat{J}_z}e^{-i\theta\hat{J}_y}e^{-i\psi\hat{J}_z}$. This takes $(0,0,|\vec{p}|)$ to $\vec{p}_{ref}$}
\label{ref_rot}
\end{table}

	\item Each momentum and particular $J^P$  can be projected into multiple allowed irrep and rows. The code finds the allowed irrep and rows and loops over them to get all the possible combinations at the source, sink and insertion. (Table: \ref{irrep_table})

\begin{table}[!htbp]
\centering
\hspace{9mm}
\begin{minipage}{0.5\linewidth}
\begin{tabular}{c|c|c}

Lattice & Little Group & Irreps($\Lambda$ or $\Lambda^p$) \\ 
Momentum & (Double Cover) & (For single cover)\\

\hline

$(0,0,0)$ & $O^{D}_h$ & $A_1^{\pm} \ A_2^{\pm} \ E^{\pm} \ T_1^{\pm} \ T_2^{\pm}$\\ 
$(0,0,n)$ & $\mbox{Dic}_4$ & $A_1 \ A_2 \ B_1 \ B_2 \ E_2 $\\
$(0,n,n)$ & $\mbox{Dic}_2$ & $A_1 \ A_2 \ B_1 \ B_2 $\\
$(n,n,n)$ & $\mbox{Dic}_3$ & $A_1 \ A_2 \ E_2 $\\

\end{tabular}
\end{minipage}
\caption{Allowed lattice momenta on a cubic lattice in a finite cubic box, along with the corresponding little groups for relevant momenta(the double covers relevant for integer and half-integer spin). We list only the single cover irreps relevant for integer spin. Lattice momenta are given in units of $2\pi/(L_s a_s )$ where $n \in Z$  is a non-zero integers. The $A$ and $B$ irreps have dimension one, $E$ two
and $T$ three. Dic$_{n}$ is the dicyclic group of order 4$n$}
\label{irrep_table}
\end{table}


	\item For this process there is a  polarization four-vector associated with  the vector quantity. The polarization for a given helicity $\lambda$ and p is determined by finding the polarization along z-axis for the given $\lambda$ and since rotations only change the direction of p leaving $\lambda$ unchanged. Using this relation $\epsilon(Rp, \lambda) = R\epsilon(p,\lambda)$ \cite{hel}. \par The convention for the polarization 3-vector in \cite{hel} is used. To convert it into a 4-vector we use the relation $\epsilon^{\mu}p_{\mu} = 0$. The rotation matrix is calculated in step:2 is used and again the convention for the 4-vector is [$\epsilon_0,-\vec{\epsilon}$]

\begin{table}[!htbp]
\centering
\hspace{9mm}
\begin{minipage}{0.33\linewidth}
\begin{tabular}{c|c}

Helicity & Polarization Vector \\ 
$\lambda$ & $\epsilon(p_z,\lambda)$ \\

\hline
$+1$ & $\frac{1}{\sqrt{2}}\Big[ 0\ , +1\ , +i\ ,0 \Big]$ \\ \\
$0$ & $\frac{1}{M}\Big[|\vec{p}| \ , 0\ ,0\ , -E \Big]$\\ \\
$-1$ & $\frac{1}{\sqrt{2}}\Big[  0\ , -1\ , +i\ ,0 \Big]$ \\


\end{tabular}
\end{minipage}
\caption{The polarization vectors for $p_z$}
\label{pol_table}
\end{table}


	\item The subduction coefficients are in general a sum over $\pm|\lambda|$ and the polarization along with the Wigner-D matrix to convert the current to helicity basis are the only other quantities that depend on $\lambda$, so the sum $\sum_{\lambda = \pm|\lambda|} S^{\Lambda \tilde{\eta}}_{\mu \lambda} \epsilon(p,\lambda)$ is taken to get a four-vector at the sink. And a sum, $\sum_{\lambda_{\gamma} = \pm|\lambda_{\gamma|}} S^{\Lambda_{\gamma} \tilde{\eta}_{\gamma} }_{\mu_{\gamma}  } D^{J}_{\lambda_{\gamma} m}(R_{\gamma})\epsilon(p_{\gamma} ,m )$ is taken for the insertion, where $R_{\gamma}$ is the rotation that takes the momentum from $p_z \rightarrow p_{\gamma}$. The subduction coefficient of PS is just 1 and always $\lambda = 0 $ but to be pedantic, it is taken verified with adat.
	\item Thus we have the ingredients that go into the master equation (Eq:\ref{fnl})
	
\begin{equation}
\label{fh}
\Big[ S^{ \Lambda_f,\tilde{\eta}_f  }_{\mu_f 0} \Big] 
\Big[ \sum_{\lambda_{\gamma} = \pm|\lambda_{\gamma}|} S^{ \Lambda_{\gamma},\tilde{\eta}_{\gamma}  }_{\mu_{\gamma} \lambda_{\gamma} } \sum_{m}D^{1*}_{m\lambda_{\gamma}}(R_\gamma)i \epsilon^{*}_{\mu}(0, m) \Big ] 
\Big[  \sum_{\lambda_i = \pm|\lambda_i| } S^{ \Lambda_i,\tilde{\eta}_i  }_{\mu_i\lambda_i} \epsilon_{\sigma}(p_i, \lambda_i) \Big]  \epsilon^{\mu\nu\rho\sigma}(p_i + p_f)_{\nu}(p_i - p_f)_{\rho}
\end{equation}

Here, the terms in square brackets are calculated based on the previous step, the anti-symmetric tensor is calculated by looking the permutations to get to (0123) (even = +1 and odd = -1). The sum and difference of the four-vectors in the first step give the last two terms.
\end{itemize}

\section{Some Explicit Calculations to Check the results from the Code}
\begin{enumerate}
\item \textbf{$\mathbf{p_i}$ and $\mathbf{p_f}$ are directed along z-axis:}\par
This means both $(p_i + p_f)_{\nu}$ and $(p_i - p_f)_{\rho}$ will have only two non-zero components i.e. 0 and 3. This means that $\mu$ and $\sigma$ will have to be either 1 or 2.
We first start with the calculation of $\epsilon^{\mu\nu\rho\sigma}\epsilon_{\sigma}(p_i, \lambda)(p_i + p_f)_{\nu}(p_i - p_f)_{\rho}$ \par
In terms of irreps  A1 and T1 r2 are $\lambda = 0$, T1 r1 is $\lambda = +1 $, T1 r3 $\lambda = -1 $ and E2 r1 E2 r2 are a linear combination. In this case  (for $\tilde{\eta} = +1$) E2 r1 is $\frac{1}{\sqrt{2}}(1(+\lambda) + 1(-\lambda))$ and E2 r2 is $\frac{1}{\sqrt{2}}(1(+\lambda) - 1(-\lambda))$ as in (Table: \ref{sub})


\begin{table}[!htbp]
\centering
\hspace{9mm}
\begin{minipage}{0.5\linewidth}
\begin{tabular}{c|c|c|c}

 & $\vec{p} = {(0,0,n),(n,n,n) }$ & $\vec{p} = (0,n,n)$ & $\vec{p} = (0,0,0)$\\ 
\hline
$\lambda = 0$ & $A_1 / A_2$ & $A_1 / A_2$& $T_1(2),A_2/A1$\\
$\lambda = 1$ & $(E_2(1) + E_2(2))/\sqrt{2}$ & $(B_1 + B_2)/\sqrt{2}$& $T_1(1)$\\
$\lambda = -1$ &$(E_2(1) - E_2(2))/\sqrt{2}$ & $(B_1 - B_2)/\sqrt{2}$& $T_1(3)$\\\
\end{tabular}
\end{minipage}
\caption{Subduction Coefficients for $0^{-1}$ (only $\lambda =$ 0, A2(in motion) A1(at rest)) and $1^{-1}$}
\label{sub}
\end{table}

\begin{enumerate}
\item when $\lambda_\gamma$ = 0: The non-zero elements are $\mu$ = 0/3 since $\nu(\rho)$ = 0/3, this is zero
\item when $\lambda_i$ = 0:  Similar to above the element is zero.
\end{enumerate}
\par Thus, whenever there is T1 r2 or A1, the and the momenta are along the z-axis the pre-factor is always 0. When $\lambda_\gamma = \pm 1$ and $\lambda_i = \pm 1$,
\begin{enumerate}
\item $\mu$ = 0:  0 for all $\lambda_\gamma$ and $\lambda_i$ because of the statement in the previous paragraph.
\item $\mu$ = 3:  0 for all $\lambda_\gamma$ and $\lambda_i$ because of the statement in the previous paragraph.
\end{enumerate}
When both $\lambda_\gamma = \lambda_i = \pm 1$
\begin{enumerate}
\item $\mu$ = 1:  $$\frac{i}{\sqrt{2}} \frac{1}{\sqrt{2}} \epsilon^{1\nu\rho\ 2} (p_i + p_f)_{\nu}(p_i - p_f)_{\rho}$$
\item $\mu$ = 2:  $$\frac{1}{\sqrt{2}} \frac{i}{\sqrt{2}}\epsilon^{2\nu\rho\ 1} (p_i + p_f)_{\nu}(p_i - p_f)_{\rho} $$
\end{enumerate}
The minus sign in $\frac{i}{\sqrt{2}} $ etc. are cancelled by making the polarization a four vector with convention (1,-1,-1,-1). The vector that transforms to circular basis is $-i*\epsilon$ thus there will be an additional $-i$ factor as well. Also, for the the z-axis the Wigner-D matrix is 1 and $\lambda_{\gamma} = m$ which is not explicitly shown. \par
Adding (a) and (b) shows
$$\epsilon^{\mu\nu\rho\sigma}\epsilon_{\mu}(p_i - p_f, \lambda_{\gamma} = \pm 1)\epsilon_{\sigma}(p_i, \lambda_i = \pm 1)(p_i + p_f)_{\nu}(p_i - p_f)_{\rho} = 0$$
Thus, when both insertion and initial states are T1 r1, T1 r3, E2 r1 or E2 r2 the pre-factor is zero. In the case of E2 also the linear combinations cancel out. Also when both V and Current are in T1 irrep both the momenta are 0 and thus the pre-factor is 0. \par


\begin{table}[!htbp]
\caption {Table of Zeros for $p_z$}
\begin{minipage}{0.33\textwidth}
\raggedright
\begin{tabular}{cc c}
 &$p_z \rightarrow p_z' $&\\ \hline
 
&$ D4A1/A2 \rightarrow D4T1 r1 \rightarrow D4T1 r1 $& 0\\
&$ D4A1/A2 \rightarrow D4T1 r2 \rightarrow D4T1 r1 $& 0\\
&$ D4A1/A2 \rightarrow D4T1 r3 \rightarrow D4T1 r1 $& 0\\
&$ D4A1/A2 \rightarrow D4A1 \rightarrow D4T1 r1 $& 0\\
&$ D4A1/A2 \rightarrow D4E2 r1 \rightarrow D4T1 r1 $& x\\
&$ D4A1/A2 \rightarrow D4E2 r2 \rightarrow D4T1 r1 $& x\\ \hline
&$ D4A1/A2 \rightarrow D4T1 r1 \rightarrow D4T1 r2 $& 0\\
&$ D4A1/A2 \rightarrow D4T1 r2 \rightarrow D4T1 r2 $& 0\\
&$ D4A1/A2 \rightarrow D4T1 r3 \rightarrow D4T1 r2 $& 0\\
&$ D4A1/A2 \rightarrow D4A1 \rightarrow D4T1 r2 $& 0\\
&$ D4A1/A2 \rightarrow D4E2 r1 \rightarrow D4T1 r2 $& 0\\
&$ D4A1/A2 \rightarrow D4E2 r2 \rightarrow D4T1 r2 $& 0\\ \hline

\end{tabular}
\end{minipage}
\begin{minipage}[!htbp]{0.33\textwidth}
\centering
\begin{tabular}{ccc}
&$p_z \rightarrow p_z' $&\\ \hline

&$ D4A1/A2 \rightarrow D4T1 r1 \rightarrow D4T1 r3 $& 0\\
&$ D4A1/A2 \rightarrow D4T1 r2 \rightarrow D4T1 r3 $& 0\\
&$ D4A1/A2 \rightarrow D4T1 r3 \rightarrow D4T1 r3 $& 0\\
&$ D4A1/A2 \rightarrow D4A1 \rightarrow D4T1 r3 $& 0\\
&$ D4A1/A2 \rightarrow D4E2 r1 \rightarrow D4T1 r3 $& x\\
&$ D4A1/A2 \rightarrow D4E2 r2 \rightarrow D4T1 r3 $& x\\ \hline
&$ D4A1/A2 \rightarrow D4T1 r1 \rightarrow D4A1 $& 0\\
&$ D4A1/A2 \rightarrow D4T1 r2 \rightarrow D4A1 $& 0\\
&$ D4A1/A2 \rightarrow D4T1 r3 \rightarrow D4A1 $& 0\\
&$ D4A1/A2 \rightarrow D4A1 \rightarrow D4A1 $& 0\\
&$ D4A1/A2 \rightarrow D4E2 r1 \rightarrow D4A1 $& 0\\
&$ D4A1/A2 \rightarrow D4E2 r2 \rightarrow D4A1 $& 0\\ \hline

\end{tabular}
\end{minipage}
\begin{minipage}[!htbp]{0.33\textwidth}
\raggedleft
\begin{tabular}{ccc}
&$p_z \rightarrow p_z' $&\\ \hline

&$ D4A1/A2 \rightarrow D4T1 r1 \rightarrow D4E2 r1 $& x\\
&$ D4A1/A2 \rightarrow D4T1 r2 \rightarrow D4E2 r1 $& 0\\
&$ D4A1/A2 \rightarrow D4T1 r3 \rightarrow D4E2 r1 $& x\\
&$ D4A1/A2 \rightarrow D4A1 \rightarrow D4E2 r1 $& 0\\
&$ D4A1/A2 \rightarrow D4E2 r1 \rightarrow D4E2 r1 $& 0\\
&$ D4A1/A2 \rightarrow D4E2 r2 \rightarrow D4E2 r1 $& x\\ \hline
&$ D4A1/A2 \rightarrow D4T1 r1 \rightarrow D4E2 r2 $& x\\
&$ D4A1/A2 \rightarrow D4T1 r2 \rightarrow D4E2 r2 $& 0\\
&$ D4A1/A2 \rightarrow D4T1 r3 \rightarrow D4E2 r2 $& x\\
&$ D4A1/A2 \rightarrow D4A1 \rightarrow D4E2 r2 $& 0\\
&$ D4A1/A2 \rightarrow D4E2 r1 \rightarrow D4E2 r2 $& x\\
&$ D4A1/A2 \rightarrow D4E2 r2 \rightarrow D4E2 r2 $& 0\\  \hline
\end{tabular}
\end{minipage}
\label{Tab}
\end{table}

\item \textbf{Example of a calculation not in along z-axis: }
$-1 0 0 A2 r1 \rightarrow 0 0 0 T1 r1 \rightarrow -1 0 0 E2 r2$\\\\
The 0 0 0_T1 r1 has only $\lambda_{\gamma}  = +1$ and -1 0 0_E2 r2 has $\frac{1}{\sqrt{2}}(1(+\lambda_i) - 1(-\lambda_i))$\\\\
\begin{equation}
\epsilon(p_i - p_f, \lambda_{\gamma}) \rightarrow  \epsilon(000, +1) = \frac{1}{\sqrt{2}}[0,1,i,0]
\end{equation}
\begin{equation}
\epsilon(p_i, \lambda) \rightarrow R(001 \rightarrow -100) \epsilon(001,\pm 1)  = \frac{1}{\sqrt{2}}[0,0,-i,\mp1]
\end{equation}

\[ 
\frac{1}{\sqrt{2}}
\begin{bmatrix} 
1&0&0&0\\
0&0 & 0 & -1\\
0&0 & -1 & 0\\
0&-1 & 0 & 0
\end{bmatrix} 
\begin{bmatrix}
0\\
\pm1\\
i\\
0
\end{bmatrix} 
= \frac{1}{\sqrt{2}}
\begin{bmatrix}
0\\
0\\
-i\\
\mp 1
\end{bmatrix} 
\]
$(p_i - p_f)$ has only one non-zero component $(p_i - p_f)_0$ whereas $(p_i + p_f)$ has two non-zero components, 0 and 1 so the only one that contributes is $(p_i + p_f)_1$. 
\begin{equation}
(p_i - p_f)_0 = \sqrt{m_i^2 + p_i^2} - \sqrt{m_f^2 + p_f^2} = \sqrt{m_i^2 + \Bigg(\frac{2\pi}{L\xi}\Bigg)^2} - \sqrt{m_f^2 + \Bigg( \frac{2\pi}{L\xi}\Bigg)^2} = 0.045786 
\end{equation}
\begin{equation}
(p_i + p_f)_1 = -1 \Bigg(  (-1) \Bigg(\frac{2\pi}{L\xi}\Bigg) + (-1) \Bigg( \frac{2\pi}{L\xi} \Bigg ) \Bigg) = 0.22819
\end{equation}
\begin{equation}
\epsilon(000, +1)_2 = \frac{1}{\sqrt{2}}i
\end{equation}
\begin{equation}
\frac{1}{\sqrt{2}}(\epsilon(-200, +1)_ 3 - \epsilon(-200, -1)_ 3)= \frac{1}{\sqrt{2}}(-2) \frac{1}{\sqrt{2}} = -1
\end{equation}
So, the value is $-0.00738i \times -i$, (the $-i$ from the convention of the vector used to convert to helicity basis from the m basis) which exactly matches with the result of the code.
\end{enumerate}
These results are {\color{red}consistent} with the code as tabulated in (Table: \ref{Tab}) for all z-momenta.



\begin{table}[!htbp]
\caption {The list of the non-zero Kinematic Factors}
\begin{minipage}{0.33\textwidth}
\raggedright
\begin{tabular}{cc c}
 &$p000 \rightarrow p000 $&\\ \hline
 
 & $T1 r1 \rightarrow T1 r1 \rightarrow A1$ & 0 \\
 & $T1 r2 \rightarrow T1 r1 \rightarrow A1$ &  0\\
 & $T1 r3 \rightarrow T1 r1 \rightarrow A1$ &  0\\
 & $T1 r1 \rightarrow T1 r2 \rightarrow A1$ &  0\\
 & $T1 r2 \rightarrow T1 r2 \rightarrow A1$ &  0\\
 & $T1 r3 \rightarrow T1 r2 \rightarrow A1$ &  0\\
 & $T1 r1 \rightarrow T1 r3 \rightarrow A1$ &  0\\
 & $T1 r2 \rightarrow T1 r3 \rightarrow A1$ &  0\\
 & $T1 r3 \rightarrow T1 r3 \rightarrow A1$ &  0\\

\hline

 &$p001 \rightarrow p001 $&\\ \hline

 & $D4A1 r1 \rightarrow T1 r1 \rightarrow D4A2$ & 0\\
 & $D4E2 r1 \rightarrow T1 r1 \rightarrow D4A2$ & x\\
 & $D4E2 r2 \rightarrow T1 r1 \rightarrow D4A2$ & x\\
 & $D4A1 r1 \rightarrow T1 r2 \rightarrow D4A2$ & 0\\
 & $D4E2 r1 \rightarrow T1 r2 \rightarrow D4A2$ & 0\\
 & $D4E2 r2 \rightarrow T1 r2 \rightarrow D4A2$ & 0\\
 & $D4A1 r1 \rightarrow T1 r3 \rightarrow D4A2$ & 0\\
 & $D4E2 r1 \rightarrow T1 r3 \rightarrow D4A2$ & x\\
 & $D4E2 r2 \rightarrow T1 r3 \rightarrow D4A2$ & x\\

\hline

 &$p011 \rightarrow p011 $& \\ \hline
 & $D2A1 r1 \rightarrow T1 r1 \rightarrow D2A2$ & 0\\
 & $D2B1 r1 \rightarrow T1 r1 \rightarrow D2A2$ &  x\\
 & $D2B2 r1 \rightarrow T1 r1 \rightarrow D2A2$ &  x\\
 & $D2A1 r1 \rightarrow T1 r2 \rightarrow D2A2$ &  0\\
 & $D2B1 r1 \rightarrow T1 r2 \rightarrow D2A2$ &  0\\
 & $D2B2 r1 \rightarrow T1 r2 \rightarrow D2A2$ &  x\\
 & $D2A1 r1 \rightarrow T1 r3 \rightarrow D2A2$ &  0\\
 & $D2B1 r1 \rightarrow T1 r3 \rightarrow D2A2$ &  x\\
 & $D2B2 r1 \rightarrow T1 r3 \rightarrow D2A2$ &  x\\

\hline

 &$p111 \rightarrow p111 $& \\ \hline
 & $D3A1 r1 \rightarrow T1 r1 \rightarrow D3A2$ & 0\\
 & $D3E2 r1 \rightarrow T1 r1 \rightarrow D3A2$ &  x\\
 & $D3E2 r2 \rightarrow T1 r1 \rightarrow D3A2$ &  x\\
 & $D3A1 r1 \rightarrow T1 r2 \rightarrow D3A2$ &  0\\
 & $D3E2 r1 \rightarrow T1 r2 \rightarrow D3A2$ &  x\\
 & $D3E2 r2 \rightarrow T1 r2 \rightarrow D3A2$ &  0\\
 & $D3A1 r1 \rightarrow T1 r3 \rightarrow D3A2$ &  0\\
 & $D3E2 r1 \rightarrow T1 r3 \rightarrow D3A2$ &  x\\
 & $D3E2 r2 \rightarrow T1 r3 \rightarrow D3A2$ &  x\\

 
\end{tabular}
\end{minipage}
\begin{minipage}[!htbp]{0.33\textwidth}
\centering
\begin{tabular}{ccc}

 &$p001 \rightarrow p002 $& \\ \hline
 & $D4A1 r1 \rightarrow D4A1 r1 \rightarrow D4A2$ & 0\\
 & $D4E2 r1 \rightarrow D4A1 r1 \rightarrow D4A2$ &  0\\
 & $D4E2 r2 \rightarrow D4A1 r1 \rightarrow D4A2$ &  0\\
 & $D4A1 r1 \rightarrow D4E2 r1 \rightarrow D4A2$ &  0\\
 & $D4E2 r1 \rightarrow D4E2 r1 \rightarrow D4A2$ &  0\\
 & $D4E2 r2 \rightarrow D4E2 r1 \rightarrow D4A2$ &  x\\
 & $D4A1 r1 \rightarrow D4E2 r2 \rightarrow D4A2$ &  0\\
 & $D4E2 r1 \rightarrow D4E2 r2 \rightarrow D4A2$ &  x\\
 & $D4E2 r2 \rightarrow D4E2 r2 \rightarrow D4A2$ &  0\\

\hline

 &$p010 \rightarrow p011 $& \\ \hline
 & $D4A1 r1 \rightarrow D4A1 r1 \rightarrow D2A2$ & 0\\
 & $D4E2 r1 \rightarrow D4A1 r1 \rightarrow D2A2$ &  0\\
 & $D4E2 r2 \rightarrow D4A1 r1 \rightarrow D2A2$ &  0\\
 & $D4A1 r1 \rightarrow D4E2 r1 \rightarrow D2A2$ &  0\\
 & $D4E2 r1 \rightarrow D4E2 r1 \rightarrow D2A2$ &  x\\
 & $D4E2 r2 \rightarrow D4E2 r1 \rightarrow D2A2$ &  0\\
 & $D4A1 r1 \rightarrow D4E2 r2 \rightarrow D2A2$ &  x\\
 & $D4E2 r1 \rightarrow D4E2 r2 \rightarrow D2A2$ &  0\\
 & $D4E2 r2 \rightarrow D4E2 r2 \rightarrow D2A2$ &  x\\

\hline

 &$p110 \rightarrow p111 $& \\ \hline
 & $D2A1 r1 \rightarrow D4A1 r1 \rightarrow D3A2$ & 0\\
 & $D2B1 r1 \rightarrow D4A1 r1 \rightarrow D3A2$ &  0\\
 & $D2B2 r1 \rightarrow D4A1 r1 \rightarrow D3A2$ &  0\\
 & $D2A1 r1 \rightarrow D4E2 r1 \rightarrow D3A2$ &  x\\
 & $D2B1 r1 \rightarrow D4E2 r1 \rightarrow D3A2$ &  x\\
 & $D2B2 r1 \rightarrow D4E2 r1 \rightarrow D3A2$ &  x\\
 & $D2A1 r1 \rightarrow D4E2 r2 \rightarrow D3A2$ &  x\\
 & $D2B1 r1 \rightarrow D4E2 r2 \rightarrow D3A2$ &  x\\
 & $D2B2 r1 \rightarrow D4E2 r2 \rightarrow D3A2$ &  x\\

\hline

 &$p000 \rightarrow p011 $&\\  \hline
 & $T1 r1 \rightarrow D2A1 r1 \rightarrow D2A2$ & 0\\
 & $T1 r2 \rightarrow D2A1 r1 \rightarrow D2A2$ &  0\\
 & $T1 r3 \rightarrow D2A1 r1 \rightarrow D2A2$ &  0\\
 & $T1 r1 \rightarrow D2B1 r1 \rightarrow D2A2$ &  x\\
 & $T1 r2 \rightarrow D2B1 r1 \rightarrow D2A2$ &  0\\
 & $T1 r3 \rightarrow D2B1 r1 \rightarrow D2A2$ &  x\\
 & $T1 r1 \rightarrow D2B2 r1 \rightarrow D2A2$ &  x\\
 & $T1 r2 \rightarrow D2B2 r1 \rightarrow D2A2$ &  x\\
 & $T1 r3 \rightarrow D2B2 r1 \rightarrow D2A2$ &  x\\
 
\end{tabular}
\end{minipage}
\begin{minipage}[!htbp]{0.33\textwidth}
\begin{tabular}{cc c}
\raggedleft

 &$p0-1-1 \rightarrow p000 $ & \\ \hline
 &  $D2A1 r1 \rightarrow D2A1 r1 \rightarrow A1$ & 0\\
 & $D2B1 r1 \rightarrow D2A1 r1 \rightarrow A1$ &  0\\
 & $D2B2 r1 \rightarrow D2A1 r1 \rightarrow A1$ &  0\\
 & $D2A1 r1 \rightarrow D2B1 r1 \rightarrow A1$ &  0\\
 & $D2B1 r1 \rightarrow D2B1 r1 \rightarrow A1$ &  0\\
 & $D2B2 r1 \rightarrow D2B1 r1 \rightarrow A1$ &  x\\
 & $D2A1 r1 \rightarrow D2B2 r1 \rightarrow A1$ &  0\\
 & $D2B1 r1 \rightarrow D2B2 r1 \rightarrow A1$ &  x\\
 & $D2B2 r1 \rightarrow D2B2 r1 \rightarrow A1$ &  0\\

\hline

 &$p100 \rightarrow p111 $ & \\ \hline
 & $D4A1 r1 \rightarrow D2A1 r1 \rightarrow D3A2$ & 0\\
 & $D4E2 r1 \rightarrow D2A1 r1 \rightarrow D3A2$ &  x\\
 & $D4E2 r2 \rightarrow D2A1 r1 \rightarrow D3A2$ &  x\\
 & $D4A1 r1 \rightarrow D2B1 r1 \rightarrow D3A2$ &  x\\
 & $D4E2 r1 \rightarrow D2B1 r1 \rightarrow D3A2$ &  x\\
 & $D4E2 r2 \rightarrow D2B1 r1 \rightarrow D3A2$ &  x\\
 & $D4A1 r1 \rightarrow D2B2 r1 \rightarrow D3A2$ &  0\\
 & $D4E2 r1 \rightarrow D2B2 r1 \rightarrow D3A2$ &  x\\
 & $D4E2 r2 \rightarrow D2B2 r1 \rightarrow D3A2$ &  x\\

\hline

 &$p000 \rightarrow p111 $ \\ \hline
 & $T1 r1 \rightarrow D3A1 r1 \rightarrow D3A2$ & 0\\
 & $T1 r2 \rightarrow D3A1 r1 \rightarrow D3A2$ &  0\\
 & $T1 r3 \rightarrow D3A1 r1 \rightarrow D3A2$ &  0\\
 & $T1 r1 \rightarrow D3E2 r1 \rightarrow D3A2$ &  x\\
 & $T1 r2 \rightarrow D3E2 r1 \rightarrow D3A2$ &  x\\
 & $T1 r3 \rightarrow D3E2 r1 \rightarrow D3A2$ &  x\\
 & $T1 r1 \rightarrow D3E2 r2 \rightarrow D3A2$ &  x\\
 & $T1 r2 \rightarrow D3E2 r2 \rightarrow D3A2$ &  0\\
 & $T1 r3 \rightarrow D3E2 r2 \rightarrow D3A2$ &  x\\

\hline

 & $p-100 \rightarrow p011 $ \\ \hline
 & $D4A1 r1 \rightarrow D3A1 r1 \rightarrow D2A2$ & 0\\
 & $D4E2 r1 \rightarrow D3A1 r1 \rightarrow D2A2$ &  0\\
 & $D4E2 r2 \rightarrow D3A1 r1 \rightarrow D2A2$ &  0\\
 & $D4A1 r1 \rightarrow D3E2 r1 \rightarrow D2A2$ &  x\\
 & $D4E2 r1 \rightarrow D3E2 r1 \rightarrow D2A2$ &  x\\
 & $D4E2 r2 \rightarrow D3E2 r1 \rightarrow D2A2$ &  x\\
 & $D4A1 r1 \rightarrow D3E2 r2 \rightarrow D2A2$ &  x\\
 & $D4E2 r1 \rightarrow D3E2 r2 \rightarrow D2A2$ &  x\\
 & $D4E2 r2 \rightarrow D3E2 r2 \rightarrow D2A2$ &  x\\

\end{tabular}
\end{minipage}
\end{table}



\begin{table}[!htbp]
\caption {The list of the non-zero Kinematic Factors}
\begin{minipage}{0.33\textwidth}
\raggedright
\begin{tabular}{cc c}

 &$p-1-1-1 \rightarrow p000 $ \\ \hline
 & $D3A1 r1 \rightarrow D3A1 r1 \rightarrow A1$ & 0\\
 & $D3E2 r1 \rightarrow D3A1 r1 \rightarrow A1$ &  0\\
 & $D3E2 r2 \rightarrow D3A1 r1 \rightarrow A1$ &  0\\
 & $D3A1 r1 \rightarrow D3E2 r1 \rightarrow A1$ &  0\\
 & $D3E2 r1 \rightarrow D3E2 r1 \rightarrow A1$ &  0\\
 & $D3E2 r2 \rightarrow D3E2 r1 \rightarrow A1$ &  x\\
 & $D3A1 r1 \rightarrow D3E2 r2 \rightarrow A1$ &  0\\
 & $D3E2 r1 \rightarrow D3E2 r2 \rightarrow A1$ &  x\\
 & $D3E2 r2 \rightarrow D3E2 r2 \rightarrow A1$ &  0\\

\hline

 &$p000 \rightarrow p002 $ \\ \hline
 & $T1 r1 \rightarrow D4A1 r1 \rightarrow D4A2$ & 0\\
 & $T1 r2 \rightarrow D4A1 r1 \rightarrow D4A2$ &  0\\
 & $T1 r3 \rightarrow D4A1 r1 \rightarrow D4A2$ &  0\\
 & $T1 r1 \rightarrow D4E2 r1 \rightarrow D4A2$ &  x\\
 & $T1 r2 \rightarrow D4E2 r1 \rightarrow D4A2$ &  0\\
 & $T1 r3 \rightarrow D4E2 r1 \rightarrow D4A2$ &  x\\
 & $T1 r1 \rightarrow D4E2 r2 \rightarrow D4A2$ &  x\\
 & $T1 r2 \rightarrow D4E2 r2 \rightarrow D4A2$ &  0\\
 & $T1 r3 \rightarrow D4E2 r2 \rightarrow D4A2$ &  x\\

\hline

 &$p00-1 \rightarrow p001 $ \\ \hline
 & $D4A1 r1 \rightarrow D4A1 r1 \rightarrow D4A2$ & 0\\
 & $D4E2 r1 \rightarrow D4A1 r1 \rightarrow D4A2$ &  0\\
 & $D4E2 r2 \rightarrow D4A1 r1 \rightarrow D4A2$ &  0\\
 & $D4A1 r1 \rightarrow D4E2 r1 \rightarrow D4A2$ &  0\\
 & $D4E2 r1 \rightarrow D4E2 r1 \rightarrow D4A2$ &  0\\
 & $D4E2 r2 \rightarrow D4E2 r1 \rightarrow D4A2$ &  x\\
 & $D4A1 r1 \rightarrow D4E2 r2 \rightarrow D4A2$ &  0\\
 & $D4E2 r1 \rightarrow D4E2 r2 \rightarrow D4A2$ &  x\\
 & $D4E2 r2 \rightarrow D4E2 r2 \rightarrow D4A2$ &  0\\

\hline

 &$p01-1 \rightarrow p011 $ \\ \hline
 & $D2A1 r1 \rightarrow D4A1 r1 \rightarrow D2A2$ & 0\\
 & $D2B1 r1 \rightarrow D4A1 r1 \rightarrow D2A2$ &  0\\
 & $D2B2 r1 \rightarrow D4A1 r1 \rightarrow D2A2$ &  x\\
 & $D2A1 r1 \rightarrow D4E2 r1 \rightarrow D2A2$ &  0\\
 & $D2B1 r1 \rightarrow D4E2 r1 \rightarrow D2A2$ &  0\\
 & $D2B2 r1 \rightarrow D4E2 r1 \rightarrow D2A2$ &  x\\
 & $D2A1 r1 \rightarrow D4E2 r2 \rightarrow D2A2$ &  x\\
 & $D2B1 r1 \rightarrow D4E2 r2 \rightarrow D2A2$ &  x\\
 & $D2B2 r1 \rightarrow D4E2 r2 \rightarrow D2A2$ &  0\\

\hline

\end{tabular}
\end{minipage}
\begin{minipage}[!htbp]{0.33\textwidth}
\begin{tabular}{cc c}
\centering

 &$p00-1 \rightarrow p001 $ \\ \hline
 & $D4A1 r1 \rightarrow D4A1 r1 \rightarrow D4A2$ & 0\\
 & $D4E2 r1 \rightarrow D4A1 r1 \rightarrow D4A2$ &  0\\
 & $D4E2 r2 \rightarrow D4A1 r1 \rightarrow D4A2$ &  0\\
 & $D4A1 r1 \rightarrow D4E2 r1 \rightarrow D4A2$ &  0\\
 & $D4E2 r1 \rightarrow D4E2 r1 \rightarrow D4A2$ &  0\\
 & $D4E2 r2 \rightarrow D4E2 r1 \rightarrow D4A2$ &  x\\
 & $D4A1 r1 \rightarrow D4E2 r2 \rightarrow D4A2$ &  0\\
 & $D4E2 r1 \rightarrow D4E2 r2 \rightarrow D4A2$ &  x\\
 & $DE2 r2 \rightarrow D4E2 r2 \rightarrow D4A2$ &  0\\


\hline

 &$p002 \rightarrow p002 $&\\  \hline
 & $D4A1 r1 \rightarrow T1 r1 \rightarrow D4A2$ & 0\\
 & $D4E2 r1 \rightarrow T1 r1 \rightarrow D4A2$ &  x\\
 & $D4E2 r2 \rightarrow T1 r1 \rightarrow D4A2$ &  x\\
 & $D4A1 r1 \rightarrow T1 r2 \rightarrow D4A2$ &  0\\
 & $D4E2 r1 \rightarrow T1 r2 \rightarrow D4A2$ &  0\\
 & $D4E2 r2 \rightarrow T1 r2 \rightarrow D4A2$ &  0\\
 & $D4A1 r1 \rightarrow T1 r3 \rightarrow D4A2$ &  0\\
 & $D4E2 r1 \rightarrow T1 r3 \rightarrow D4A2$ &  x\\
 & $D4E2 r2 \rightarrow T1 r3 \rightarrow D4A2$ &  x\\

\hline

 &$p000 \rightarrow p001 $& \\ \hline
 & $T1 r1 \rightarrow D4A1 r1 \rightarrow D4A2$ & 0\\
 & $T1 r2 \rightarrow D4A1 r1 \rightarrow D4A2$ &  0\\
 & $T1 r3 \rightarrow D4A1 r1 \rightarrow D4A2$ &  0\\
 & $T1 r1 \rightarrow D4E2 r1 \rightarrow D4A2$ &  x\\
 & $T1 r2 \rightarrow D4E2 r1 \rightarrow D4A2$ &  0\\
 & $T1 r3 \rightarrow D4E2 r1 \rightarrow D4A2$ &  x\\
 & $T1 r1 \rightarrow D4E2 r2 \rightarrow D4A2$ &  x\\
 & $T1 r2 \rightarrow D4E2 r2 \rightarrow D4A2$ &  0\\
 & $T1 r3 \rightarrow D4E2 r2 \rightarrow D4A2$ &  x\\

\hline

 &$p0-10 \rightarrow p001 $& \\ \hline
 & $D4A1 r1 \rightarrow D2A1 r1 \rightarrow D4A2$ & 0\\
 & $D4E2 r1 \rightarrow D2A1 r1 \rightarrow D4A2$ &  x\\
 & $D4E2 r2 \rightarrow D2A1 r1 \rightarrow D4A2$ &  0\\
 & $D4A1 r1 \rightarrow D2B1 r1 \rightarrow D4A2$ &  0\\
 & $D4E2 r1 \rightarrow D2B1 r1 \rightarrow D4A2$ &  x\\
 & $D4E2 r2 \rightarrow D2B1 r1 \rightarrow D4A2$ &  0\\
 & $D4A1 r1 \rightarrow D2B2 r1 \rightarrow D4A2$ &  x\\
 & $D4E2 r1 \rightarrow D2B2 r1 \rightarrow D4A2$ &  0\\
 & $D4E2 r2 \rightarrow D2B2 r1 \rightarrow D4A2$ &  x\\
 
\hline

\end{tabular}
\end{minipage}
\begin{minipage}[!htbp]{0.33\textwidth}
\begin{tabular}{cc c}
\raggedleft

 &$p0-11 \rightarrow p002 $& \\ \hline
 & $D2A1 r1 \rightarrow D2A1 r1 \rightarrow D4A2$ & 0\\
 & $D2B1 r1 \rightarrow D2A1 r1 \rightarrow D4A2$ &  0\\
 & $D2B2 r1 \rightarrow D2A1 r1 \rightarrow D4A2$ &  0\\
 & $D2A1 r1 \rightarrow D2B1 r1 \rightarrow D4A2$ &  0\\
 & $D2B1 r1 \rightarrow D2B1 r1 \rightarrow D4A2$ &  0\\
 & $D2B2 r1 \rightarrow D2B1 r1 \rightarrow D4A2$ &  x\\
 & $D2A1 r1 \rightarrow D2B2 r1 \rightarrow D4A2$ &  x\\
 & $D2B1 r1 \rightarrow D2B2 r1 \rightarrow D4A2$ &  x\\
 & $D2B2 r1 \rightarrow D2B2 r1 \rightarrow D4A2$ &  0\\
 
 
 
 \hline

 & $p-1-10 \rightarrow p001 $ \\ \hline
 & $D2A1 r1 \rightarrow D3A1 r1 \rightarrow D4A2$ & 0\\
 & $D2B1 r1 \rightarrow D3A1 r1 \rightarrow D4A2$ &  x\\
 & $D2B2 r1 \rightarrow D3A1 r1 \rightarrow D4A2$ &  0\\
 & $D2A1 r1 \rightarrow D3E2 r1 \rightarrow D4A2$ &  x\\
 & $D2B1 r1 \rightarrow D3E2 r1 \rightarrow D4A2$ &  0\\
 & $D2B2 r1 \rightarrow D3E2 r1 \rightarrow D4A2$ &  x\\
 & $D2A1 r1 \rightarrow D3E2 r2 \rightarrow D4A2$ &  0\\
 & $D2B2 r1 \rightarrow D3E2 r2 \rightarrow D4A2$ &  x\\
 & $D2B2 r1 \rightarrow D3E2 r2 \rightarrow D4A2$ &  0\\

\hline

 &$p-1-11 \rightarrow p002 $ \\ \hline
 & $D3A1 r1 \rightarrow D3A1 r1 \rightarrow D4A2$ & 0\\
 & $D3E2 r1 \rightarrow D3A1 r1 \rightarrow D4A2$ &  0\\
 & $D3E2 r2 \rightarrow D3A1 r1 \rightarrow D4A2$ &  0\\
 & $D3A1 r1 \rightarrow D3E2 r1 \rightarrow D4A2$ &  x\\
 & $D3E2 r1 \rightarrow D3E2 r1 \rightarrow D4A2$ &  0\\
 & $D3E2 r2 \rightarrow D3E2 r1 \rightarrow D4A2$ &  x\\
 & $D3A1 r1 \rightarrow D3E2 r2 \rightarrow D4A2$ &  0\\
 & $D3E2 r1 \rightarrow D3E2 r2 \rightarrow D4A2$ &  x\\
 & $D3E2 r2 \rightarrow D3E2 r2 \rightarrow D4A2$ &  0\\
 
 \hline

 &\\\hline\\\\\\\\\\\\\\\\\\\hline
 
\end{tabular}
\end{minipage}
\end{table}
 
\newpage
\section{Fixing the Phase}
When the helicity operators are rotated there is an arbitrary phase $e^{i\phi}$ associated with them as shown in the appendix of \cite{shultz}. 
\begin{equation}
\hat{R}\, \big| \vec{p}; J,\lambda \big\rangle = e^{i \Phi(R, \vec{p}, J, \lambda)}\, \big| R \vec{p}; J,\lambda \big\rangle,
\label{hel_phase}
\end{equation}
where the helicity is left invariant and where rotations about the direction of the momentum, $\vec{p}$, introduce a helicity-dependent phase. Following the derivation in \cite{shultz},
\begin{equation}
e^{i \Phi(R, \vec{p}, J, \lambda)} = D^{(J)}_{\lambda\lambda}\big(R^{-1}_{R\hat{p}} R  R_{\hat{p}} \big). \label{hel_phase_D}
\end{equation}

\begin{table}[!htbp]
\centering
\hspace{9mm}
\begin{minipage}{0.1\linewidth}
\begin{tabular}{c c }
\hline
&m \\
m \' & 0 \\
\hline
\\

0& 1\\

\hline
\end{tabular}
\end{minipage}
\hspace{9mm}
\begin{minipage}{0.5\linewidth}
\begin{tabular}{c c c c}
\hline
&&m& \\
m\' & 1 & 0 & -1\\
\hline
\\
1&$\frac{1 + cos(\beta)}{2}e^{-i(\alpha + \gamma)}$&$ -\frac{1}{\sqrt{2}}sin\beta e^{-i\alpha}$&$ \frac{1 - cos(\beta)}{2}e^{-i(\alpha - \gamma)}$\\
\\
0&$ \frac{1}{\sqrt{2}}sin\beta e^{-i\gamma}$&$ cos(\beta)$& $-\frac{1}{\sqrt{2}}sin\beta e^{i\gamma}$\\
\\
-1&$\frac{1 - cos(\beta)}{2}e^{i(\alpha - \gamma)}$&$ \frac{1}{\sqrt{2}}sin\beta e^{i\alpha}$&$ \frac{1 + cos(\beta)}{2}e^{i(\alpha + \gamma)}$\\
\\
\hline
\end{tabular}

\end{minipage}
\label{wigner}
\caption{The Wigner rotation matricies for $D^0_{0,0}(\alpha,\beta,\gamma)$ and $D^1_{m',m}(\alpha,\beta,\gamma)$, ie. rotate by $\alpha$ first.}
\end{table}
 
This means that there is an arbitrary phase in the kinematic factors calculated. This does not mean anything, but when we have the tuple of momenta that are related by rotation, they have to have the same kinematic-factor i.e. a consistent phase. But the calculations so far do not guarantee this. \par
For a rotation R,
\begin{align*}
\big\langle \vec{p}\,'; J'& ,\lambda'  \big| \, j^\mu \, \big| \, \vec{p}; J, \lambda  \big\rangle  \\
 &= \big\langle \vec{p}\,'; J' ,\lambda' \big| \, \hat{R}^{-1}\hat{R} \, j^\mu \,  \hat{R}^{-1}\hat{R} \, \big| \, \vec{p}; J, \lambda  \big\rangle \\
&= \left[R^{-1}\right]^\mu_\nu    \big\langle \vec{p}\,'; J' ,\lambda'  \big|\,  \hat{R}^{-1} \, j^\nu\,  \hat{R} \, \big| \, \vec{p}; J, \lambda  \big\rangle,\\
&=  \left[R^{-1}\right]^\mu_\nu    e^{-i \Phi(R, \vec{p}\,, J', \lambda')} e^{i \Phi(R, \vec{p}, J, \lambda)} \big\langle \hat{R}\vec{p}\,'; J' ,\lambda'  \big|\,  \, j^\nu\,  \, \big| \, \hat{R}\vec{p}; J, \lambda  \big\rangle
\end{align*}
\iffalse{\color{blue} [to be deleted... ] \par When the current operator is also a helicity operator, it transforms the same way as that of a helicity state \cite{hel}. Thus, under a rotation R,
\begin{align}
\big\langle \vec{p}\,'; J'& ,\lambda'  \big| \, O_{J_\gamma}^{\dagger}(\vec{p}\,' - \vec{p},\lambda_{\gamma}) \, \big| \, \vec{p}; J, \lambda  \big\rangle  \\
 &=    e^{-i \Phi(R, \vec{p}\,, J', \lambda')} e^{i \Phi(R, \vec{p}, J, \lambda)} \big\langle \hat{R}\vec{p}\,'; J' ,\lambda'  \big|\,  \, \hat{R}^{-1} O_{J_\gamma}^{\dagger}(\vec{p}\,' - \vec{p},\lambda_{\gamma}) \hat{R} \,  \, \big| \, \hat{R}\vec{p}; J, \lambda  \big\rangle \\
&=    e^{-i \Phi(R, \vec{p}\,, J', \lambda')} e^{i \Phi(R, \vec{p}\,' - \vec{p}\,, J_{\gamma}, \lambda_{\gamma})}  e^{i \Phi(R, \vec{p}, J, \lambda)} \big\langle \hat{R}\vec{p}\,'; J' ,\lambda'  \big|\,  \, O_{J_\gamma}^{\dagger}(\hat{R}(\vec{p}\,' - \vec{p}),\lambda_{\gamma}) \,  \, \big| \, \hat{R}\vec{p}; J, \lambda  \big\rangle
\label{ph}
\end{align}
where the phase is defined by (Eq: \ref{hel_phase_D})}\fi \par
In the equation above, the current is in a cartesian four-vector basis. But, in redstar the current is in helicity basis. We have to convert the current from cartesian to circular basis and then to helicity basis using the convention in  \cite{hel},
\begin{align}
\big\langle \vec{p}\,'; J'& ,\lambda'  \big| \, j(\vec{p}\,' - \vec{p},\lambda_{\gamma}) \, \big| \, \vec{p}; J, \lambda  \big\rangle  \\
&= \sum_{m}D^{1*}_{m\lambda_{\gamma}}(R_\gamma)\epsilon^{*}_{\mu}(0, m)  \left[R^{-1}\right]^\mu_\nu    e^{-i \Phi(R, \vec{p}\,, J', \lambda')} e^{i \Phi(R, \vec{p}, J, \lambda)} \big\langle \hat{R}\vec{p}\,'; J' ,\lambda'  \big|\,  \, j^\nu\,  \, \big| \, \hat{R}\vec{p}; J, \lambda  \big\rangle
\label{phf}
\end{align}
We calculate (Eq: \ref{phf}) in the code to get the kinematic factors. Note that the sum over $\mu$ is the sum over three vector. To give $\lambda_{\gamma} = \pm 1, 0$. The spatial part of the current is subduced separately in a $1D$ irrep and this is separation is important in an anisotropic lattice. \par

\subsection{Dealing with Phases in the code}
In order to fix the phases for a group of tuple of momenta related to each other by rotations,
\begin{itemize}
 	\item We rotate the momentum of the Vector(V) to the canonical momentum. This is an allowed lattice rotation.
	\item We find the rotation matrix for this rotation which is essentially $R_{lat}^{-1}$ in \cite{hel}.

\begin{equation}
R_{lat}R_{ref}p_z = p \hspace{5mm}and\hspace{5mm} R_{ref}p_z = p_{can} \hspace{5mm}so\hspace{5mm} R_{lat}p_{can} = p
\label{rot_mat}
\end{equation}
	
	\item $\bra{PS}C\ket{V} = (\bra{PS}R^{-1})(RCR^{-1})(R\ket{V}) $, so we use $R_{lat}^{-1}$ to rotate the current and the PS to their new momenta. The current momentum is cross checked using conservation of momentum.
	\item The value of the kinematic factor for the matrix element with the V in the reference momentum and the other momenta transformed using the same rotation matrix is calculated along with a phase factor that is unique to this matrix element and depends on $\lambda$ of the current and V following the derivation in \cite{shultz}. \par
	Using (Eq:\ref{ph}) and applying it in  (Eq:\ref{fh})

\begin{align}
\big\langle \vec{p}\ ,'; J'& ,\lambda'  \big| \, O_{J_\gamma}^{\dagger}(\vec{p}\,' - \vec{p}\ ,\lambda_{\gamma}) \, \big| \, \vec{p}; J, \lambda  \big\rangle = \\
&\Big[ e^{-i \Phi(R, \vec{p}_f\,, J_f, \lambda_f)} S^{ \Lambda_f,\tilde{\eta}_f  }_{\mu_f 0} \Big] 
\Big[ \sum_{\lambda_{\gamma} = \pm|\lambda_{\gamma}|} i \left[R^{-1}\right]^\mu_\nu S^{ \Lambda_{\gamma},\tilde{\eta}_{\gamma}  }_{\mu_{\gamma} \lambda_{\gamma} }\sum_{m}D^{1*}_{m\lambda_{\gamma}}(R_\gamma)\epsilon^{*}_{\mu}(0, m)\Big ] \\
&\Big[  \sum_{\lambda_i = \pm|\lambda_i| } e^{i \Phi(R, \vec{p}_i, J_i, \lambda_i)} S^{ \Lambda_i,\tilde{\eta}_i  }_{\mu_i\lambda_i} \epsilon_{\sigma}(\hat{R}p_i, \lambda_i) \Big]  \epsilon^{\mu\nu\rho\sigma}(\hat{R}p_i + \hat{R}p_f)_{\nu}(\hat{R}p_i - \hat{R}p_f)_{\rho}
\label{final_ph}
\end{align}
We use (Eq:\ref{final_ph}) to calculate the factors and fix the phases for all the momenta related by rotations to the same value.	
\end{itemize}

\section{Some Explicit Calculations to check the Phasing}
For the PS ($J = 0\ ,\lambda = 0$ ) and $e^{-i \Phi(R, \vec{p}_f\,, J_f, \lambda_f)}$ is $1$. When $\lambda = 0$ we don't have to sum over $\pm \lambda$, but when $\lambda = \pm 1$, one has to sum. It is not clear how in the case of $\lambda \ne 0$ the overall factor turns out to be just a phase as the phase for $\lambda = \pm 1$ is $\frac{1 + cos(\beta)}{2}e^{\mp i(\alpha + \gamma)}$
\begin{enumerate}
	\item For the PS ($J = 0\ ,\lambda = 0$ ) phase is $1$. 

	\item The phase is $D^{(J)}_{\lambda\lambda}\big(R^{-1}_{R\hat{p}} R  R_{\hat{p}} \big)$.
\begin{align*}
R_{\hat{p}}p_z = p; \hspace{5mm}  
&R_{R\hat{p}} p_z = Rp = R(R_{\hat{p}}p_z ) ; \hspace{5mm} 
R_{R\hat{p}} = RR_{\hat{p}}\\
&R^{-1}_{R\hat{p}} R  R_{\hat{p}} = (RR_{\hat{p}})^{-1}R_{R\hat{p}} = 1
\end{align*}
Thus, if the rotation `R' is defined with respect to the canonical momentum of the Vector, the phase of the vector is always 1.
	\item The (Eq:\ref{final_ph}) becomes,
\begin{align*}
\big\langle \vec{p}\ ,'; J'& ,\lambda'  \big| \, O_{J_\gamma}^{\dagger}(\vec{p}\,' - \vec{p}\ ,\lambda_{\gamma}) \, \big| \, \vec{p}; J, \lambda  \big\rangle = \\
&\Big[  S^{ \Lambda_f,\tilde{\eta}_f  }_{\mu_f 0} \Big] 
\Big[ \sum_{\lambda_{\gamma} = \pm|\lambda_{\gamma}|} i \left[R^{-1}\right]^\mu_\nu S^{ \Lambda_{\gamma},\tilde{\eta}_{\gamma}  }_{\mu_{\gamma} \lambda_{\gamma} } \sum_{m}D^{1*}_{m\lambda_{\gamma}}(R_\gamma)\epsilon^{*}_{\mu}(0, m)\Big ] \\
&\Big[  \sum_{\lambda_i = \pm|\lambda_i| } S^{ \Lambda_i,\tilde{\eta}_i  }_{\mu_i\lambda_i} \epsilon_{\sigma}(\hat{R}p_i, \lambda_i) \Big]  \epsilon^{\mu\nu\rho\sigma}(\hat{R}p_i + \hat{R}p_f)_{\nu}(\hat{R}p_i - \hat{R}p_f)_{\rho}
\label{final_ph1}
\end{align*}	
\end{enumerate}
The $R_\gamma$ used above is the rotation from the z-axis to $p_i - p_f$, i.e. the current momentum before the rotation. Surprisingly, it is seen that the values of the kinematic factors do not change at all with or without accounting for the arbitrary phase associated with the helicity basis. \par




\iffalse {\color{blue} [to be deleted...]\par
\begin{enumerate}
	\item When the irrep at the insertion is subduced to T1 r2 or A2, the $\lambda = 0$, there is not sum ober $\lambda$ and there is just an overall factor. From (Table: \ref{wigner}), it is $cos(\beta)$ and for this to be a phase, $\beta = 0/\pi$. And also $cos(\beta)$ has to $+1$ for the overall normalization in the case of $\lambda = \pm 1$ to be 1. After looking at the output of the code, $\beta = 0$ always as required.
	
	\item When the irrep has a linear combination of $\lambda \pm 1$,
\begin{align*}
\sum_{\lambda = \pm 1} S^{ \Lambda,\tilde{\eta}  }_{\mu \lambda} \epsilon(p , \lambda) 
&= \sum_{\lambda = \pm 1} e^{i \Phi(R, \vec{p}\, , J = 1, \lambda)} S^{ \Lambda,\tilde{\eta}  }_{\mu \lambda} \epsilon(\hat{R}p , \lambda)
\end{align*}
The subduction coefficient for irreps where both $\pm 1$ contribute has $S^{ \Lambda,\tilde{\eta}  }_{\mu \lambda} = \pm S^{ \Lambda,\tilde{\eta}  }_{\mu -\lambda}$. As is clear from (Table: \ref{sub}). Using this and the fact $e^{i \Phi(R, \vec{p}\, , J = 1, \pm 1)} = \frac{1 + cos(\beta)}{2}e^{\mp i(\alpha + \gamma)}$ for $R(\alpha,\beta,\gamma)$. The normalization for the subduction coefficient is $\frac{1}{\sqrt{2}}$.
\begin{align*}
\sum_{\lambda = \pm 1} S^{ \Lambda,\tilde{\eta}  }_{\mu \lambda} \epsilon(p , \lambda) 
&= \frac{1 + cos(\beta)}{2\sqrt{2}}e^{- i(\alpha + \gamma)}\epsilon(\hat{R}p , +1) \pm \frac{1 + cos(\beta)}{2}e^{ i(\alpha + \gamma)}\epsilon(\hat{R}p , -1)\\
&= \frac{1 + cos(\beta)}{2\sqrt{2}}\Big[ e^{- i(\alpha + \gamma)}\epsilon(\hat{R}p , +1) \pm e^{ i(\alpha + \gamma)}\epsilon_{\mu}(\hat{R}p , -1) \Big]\\
&= \frac{1 + cos(\beta)}{2\sqrt{2}}\hat{R}R_{z} \Big[ e^{- i(\alpha + \gamma)}\epsilon(p_z , +1) \pm e^{ i(\alpha + \gamma)}\epsilon(p_z , -1) \Big]\\
&= \frac{1 + cos(\beta)}{4}\hat{R}R_{z} \Big[ 0\ , e^{- i(\alpha + \gamma)} \mp e^{ i(\alpha + \gamma)}\ , ie^{- i(\alpha + \gamma)} \pm ie^{ i(\alpha + \gamma)}\ ,0 \Big]
\end{align*}
The real part of the `phase' is the same for $+\lambda$ and $-\lambda$ but the imaginary part is the negative of each other as is clear from the expression above. This cannot be a  phase unless for special cases where the imaginary part is 0 and $\beta = 0$ for the overall factor to be 1. \par
Looking at the code, there are cases where the imaginary part in non-zero but $\beta = 0$ always. This makes sense as in the continuum where the $\lambda$s do not mix, this factor is a complex phase but when subduced, in certain irreps they have complex conjugate factors multiplying the $\pm \lambda$ and they change the value of the kinematic factor significantly. \par
In some cases where the current is in the $T1$ irrep and only $\lambda = 0$ contributes, the irrep rotates into an irrep that has a kinematic-factor of 0. The rotations are valid. Not sure about the reason.
\end{enumerate}}\fi


\newpage
\begin{thebibliography}{9}

\bibitem{hel} 
     \textit{Phys. Rev. D 85, 014507},         
       Christopher E. Thomas, Robert G. Edwards, and Jozef J. Dudek    

\bibitem{shultz} 
     \textit{Phys.Rev. D91, 114501 (2015)},         
       C. J. Shultz, J. J. Dudek, and R. G. Edwards       


\end{thebibliography}

\end{document}